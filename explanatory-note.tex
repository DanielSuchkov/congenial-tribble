\documentclass[12pt, a4paper]{fallen_report}

\author{
    Группа: \textbf{АВТ-415}\\
    Студент: Сучков Д.В.\\
    Преподаватель: \textbf{Новицкая Ю.В.}
}

\date{
    НОВОСИБИРСК 2015
}

\title{
    Пояснительная записка\\к курсовой работе\\по дисциплине «Программирование»
}

\organisation {
    МИНИСТЕРСТВО ОБРАЗОВАНИЯ И НАУКИ РОССИЙСКОЙ ФЕДЕРАЦИИ\\
    ФЕДЕРАЛЬНОЕ ГОСУДАРСТВЕННОЕ БЮДЖЕТНОЕ\\
    ОБРАЗОВАТЕЛЬНОЕ УЧРЕЖДЕНИЕ ВЫСШЕГО ПРОФЕССИОНАЛЬНОГО ОБРАЗОВАНИЯ\\
    НОВОСИБИРСКИЙ ГОСУДАРСТВЕННЫЙ ТЕХНИЧЕСКИЙ УНИВЕРСИТЕТ
}

\makeindex

\begin{document}
\maketitle
\tableofcontents
\pagebreak

\section{Задание}
Реализовать АТД «Файловая структура данных "Хешированный файл"». Метод хеширования – метод цепочек. Записи закрепленные. Ключ записи – строка символов переменной длины. Структура хеш-таблицы страничная.

\pagebreak
\section{Описание основных классов}
\subsection{Шаблон класса HashedFile}
\par Основным является шаблон класса {\consolas HashedFile}, который позволяет пользователю выбрать на этапе компиляции тип ключа, значения и размер страницы хеш-таблицы. Требования к типу ключа и значения – они должны иметь переопределенный оператор $>>$ и $<<$ для классов {\consolas fcl::BinIStreamWrap} и {\consolas fcl::BinOStreamWrap} соответственно, либо должны иметь тривиальный конструктор копирования, тогда будет использован оператор по умолчанию.

Данный класс содержит в себе две структуры данных: экземпляр класса \linebreak{\consolas FileHashIndex<Key, streamoff, PageLength>}, являющийся ассоциативным контейнером, который используется для ассоциирования ключа (тип которого определяет параметр {\consolas Key}) с файловым указателем на значения (тип значения определяется параметром {\consolas Value}) и экземпляр класса {\consolas FileStorage<Value>}, который предоставляет доступ к значениям записей по файловому указателю, полученному из хеш-таблицы. Т.е. {\consolas FileHashIndex} представляет индекс к хранилищу {\consolas FileStorage}.\\

{ \consolas\scriptsize\hskip -40pt\input{Diagram1.tex} }\\
\textit{\small На данной схеме отображены только ключевые поля и методы.}\\
Из-за следования принципу SOLID, внутреннее устройство данного класса предельно простое.
\pagebreak

Единственный примечательный метод – получение элемента по ключу:
\begin{lstlisting}[caption=, language=c++]
boost::optional<Value> get(const Key &key) const 
{
    auto pos_opt = m_index.get(key);
    if (pos_opt.is_initialized()) 
    {
        return m_storage.get(pos_opt.get());
    }
    else
    {
        return boost::none;
    }
}
\end{lstlisting}

В зависимости от того, вернёт ли хеш-таблица адрес значения, она возвращает либо значение и за {\consolas FileStorage}, либо {\consolas boost::none}, что позволяет прозрачным для пользователя образом выразить возможность отсутствие значения.

\pagebreak
\subsection{Шаблон класса FileHashIndex}
Данный шаблон реализует структуру данных <<хеш-таблица>> во внешней памяти. Тип ключа произвольный, определяется шаблонным параметром {\consolas Key}, тип значения определяется шаблонным параметром {\consolas Value}, требования к нему – должен иметь тривиальный конструктор копирования. Коллизии в хеш-таблице разрешаются методом цепочек. Каждая цепочка представлена одним или несколькими экземплярами структуры {\consolas FileHashIndex::Page} (в терминологии проекта – страница). Каждая страница хранит фрагмент цепочки, длинна фрагмента регулируется шаблонным параметром {\consolas PageLength}. Каждое звено цепочки представлено экземпляром {\consolas FileHashIndex::Page::Segment}, который хранит состояние звена (необходимо из-за того, что записи закрепленные), хеш-сумму ключа, значение и файловый указатель на сам ключ. Решение хранить ключи в отдельном файле, а в таблице хранить только файловые указатели на ключи принято для того, чтобы можно было в качестве ключа использовать любой тип данных (например, строки), а не только с тривиальным конструктором копирования, иначе нельзя было бы точно определить, какой размер занимает в файле одно звено. 

Для взаимодействия с файловыми потоками используется класс {\consolas fcl::BinIOStreamWrap}, который представляет из себя "обёртку" над любым потоком. За счёт этого инкапсулируется бинарный ввод/вывод, синтаксис ввода вывода в двоичном формате соответствует обычному вводу/выводу для стандартных потоков.

При инициализации, если пользователь даёт указание на перезапись, файл индекса заполняется двумя пустыми страницами, которые являются началом первой и второй цепочки соответственно. Если же пользователь даёт указание продолжить работу с уже существующей базой, производится только считывание основных параметров предыдущей базы (количество элементов, размер одной страницы и количество цепочек). В случае, если размер страницы не совпадает с размером страницы текущего инстанса FileHashIndex, пользователю будет выброшено исключение IncompatableFormat.
\begin{lstlisting}[caption=, language=c++]
FileHashIndex(
        const fs::path &table_path,
        const fs::path &keys_path,
        const bool overwrite)
    : m_table_path(table_path.string())
    , m_keys_path(keys_path.string()) 
{
    init_keys(overwrite);
    init_table(2, overwrite);
}

...

void init_table(const uint64_t initial_bucket_count, const bool overwrite) 
{
    try_to_open(m_table_path, m_table_file, overwrite);

    if (overwrite) 
    {
        m_bucket_count = initial_bucket_count;
        m_table.goto_begin();
        m_table << m_bucket_count << m_size << PageLength;

        // init a number of empty buckets
        for (uint64_t i = 0; i < initial_bucket_count; ++i) 
        {
            m_table << Page::get_empty();
        }
    }
    else 
    {
        m_table.goto_begin();
        m_table >> m_bucket_count >> m_size;
        auto pageLength = fcl::read_val<uint64_t>(m_table);
        if (pageLength != PageLength) 
        {
            throw IncompatableFormat();
        }
    }
}
\end{lstlisting}

Для инициализации хранилища ключей просто производится открытие файла с флагами, соответствующими запросу пользователя (перезапись/открытие существующего).

За все операции, связанные с поиском существующего элемента (проверка наличия, получение, удаление) отвечает функция inspect, которая в качестве одного из параметров принимает функтор и даёт ему указатель на значение элемента, либо {\consolas nullptr}, в случае если такого не нашлось.

\begin{lstlisting}[caption=, language=c++]
template <typename F> // Functor: Fn<auto (data_t *rec)>
auto inspect(const key_t &key, const hash_t &hash, F f) 
{
    auto page_pos = get_bucket_pos(hash);
    Page current_page;
    auto nothing = [&f] () { return f(static_cast<Segment *>(nullptr)); };
    while (true) 
    {
        m_table.set_pos(page_pos);
        m_table >> current_page;
        assert(current_page.seg_count <= PageLength);
        for (size_t i = 0; i < current_page.seg_count; ++i) 
        {
            Segment &seg = current_page.segs[i];
            // if it's not alive, just continue searching
            if (seg.state != seg_state::alive) 
            {
                continue; 
            }
            if (seg.hash == hash) 
            {
                if (get_key(seg.key_adress) == key) 
                {
                    // to remember modifications
                    auto write_at_exit = wheels::finally(
                        [&]() { m_table.write_at(page_pos, current_page); }
                    );
                    return f(&seg);
                }
            }
        }
        if (current_page.next_page_pos == 0) 
        { 
            return nothing(); 
        }
        page_pos = current_page.next_page_pos;
    }
    assert(!"unreaceble code!");
    return nothing();
}
\end{lstlisting}

Функция вычисляет позицию начала нужной цепочки в файле, потом считывает первую страницу и ищет среди "живых" записей совпадение по хеш-сумме на ней. Если совпадение есть, то дополнительно сверяются сами ключи (для этого они хранятся в отдельном файле), таким образом производится защита от коллизий хеш-сумм, далее вызывается функтор с указателем на найденную запись. Если страница кончается до окончания поиска, то проверяется наличие следующей за ней страницы и поиск продолжается на ней (либо, если её нет, он кончается).

Данная функция имеет константный и не константный варианты, отличаются они тем, что не константный вариант перезаписывает страницу с записью, так что пользователь может модифицировать запись. За счёт этого, например реализована функция удаления записи.

\begin{lstlisting}[caption=, language=c++]
bool erase(const key_t &key) {
    // to erase just turn `state` to `dead` and decrease counter
    return inspect(
        key,
        m_hasher(key),
        [this](Segment *seg) {
            if (seg) {
                seg->state = seg_state::dead;
                m_size--;
                return true;
            }
            return false;
        }
    );
}
\end{lstlisting}

В качестве результата для функции получения значения ({\consolas get}) было решено использовать {\consolas boost::optional<Value>}, так как этот тип позволяет совершенно семантически прозрачно отразить в сигнатуре функции возможность отсутствия возвращаемого значения.

\begin{lstlisting}[caption=, language=c++]
boost::optional<Value> get(const Key &key) const 
{
    return inspect(
        key,
        m_hasher(key),
        [] (auto *data) -> opt_data_t 
        {
            if (data) { return { data->value }; }
            else { return boost::none; }
        }
    );
}
\end{lstlisting}

В сигнатуре приватной функции {\consolas insert} можно заметить сразу несколько не очевидных решений:
\begin{lstlisting}[caption=, language=c++]
using key_info_t = std::pair<hash_t, pos_t>;
using key_variant_t = boost::variant<key_t, key_info_t>;
...
bool insert(
        key_variant_t key,
        std::function<data_t ()> value,
        state_t initial_state)
\end{lstlisting}

Во-первых вместо самого ключа передаётся {\consolas boost::variant}, который может принимать либо значение самого ключа, либо пары (хеш-сумма ключа, адрес ключа). Сделано это для того, чтобы можно было реализовать рехеш при помощи этой функции. Ведь при рехеше, во-первых, не требуется сохранять ключ, а во-вторых, уже известна хеш-сумма ключа, значит можно второй раз её не считать, что для некоторых случаев может быть хоть и не критично, но всё же достаточно важно.\\
Вторым примечательным решением является то, что функция получает не само значение, а функцию, которая его вернёт. Это сделано потому, что операция вставки не всегда оканчивается успехом, а значит и значение ей не всегда нужно, а иногда бывает важно выдавать значение только если операция точно завершится успехом. Примером является класс {\consolas HashedFile}: в хеш-таблице он хранит адреса значений, значит если вставка не удалась, не нужно сохранять значение в {\consolas details::FileStorage}. В данном случае сохранение значения (и взятие его адреса) будет произведено только если вставка производится успешно.

Сама вставка достаточно тривиальна за исключением следующего решения: 
\begin{lstlisting}[caption=, language=c++]
Segment &seg = current_page.segs[i];
if (seg.hash == hash) 
{
    auto keys_eq = cmp_keys_visitor(seg.key_adress, m_keys);
    // if we already have one with such key, let's resurrect it
    if (boost::apply_visitor(keys_eq, key)) 
    {
        if (seg.state == seg_state::dead) 
        { // resurrection
            // other data are the same
            seg.value = value();
            seg.state = initial_state;
            m_table.write_at(page_pos, current_page);
            return true;
        }
        return false;
    }
}
\end{lstlisting}
В случае, если найдена запись с точно таким же ключом, производится проверка её состояния и если она была до этого удалена, её заменяет новая запись, при этом перезаписывается только значение и состояние. Ключ при этом также вторично не сохраняется.

\pagebreak
\section{Пользовательский интерфейс}
При запуске программы, пользователь получает сообщение {\consolas "What do you want to do? "help" to see avialable commands"}, после чего программа ожидает ввода команды пользователем. Список доступных команд может быть получен, как гласит это сообщение, при вводе команды {\consolas help}.

Список всех доступных команд следующий: 
\begin{enumerate}
\item {\consolas close\_db} – закрывает текущую базу данных. Так как используется файловая структура данных, никаких особых действий для сохранения пользовательских данных не требуется, поэтому подтверждений не запрашивается

\item {\consolas create\_db} – запрашивает у пользователя директорию и создаёт в ней новую базу данных. Если в указанной директории база данных уже имеется, она будет перезаписана

\item {\consolas exit} – завершает работу программы. Подтверждение не запрашивается по причине указанной в п.1

\item {\consolas get} – запрашивает у пользователя значение ключа и далее выполняет поиск ассоциированного с ним значения. Если такое значение не найдено, будет выведено соответствующее сообщение об ошибке. В случае, если активной базы данных нет, выводит пользователю соответствующее сообщение об ошибке

\item {\consolas has} – запрашивает у пользователя значение ключа и проверяет, есть ли в базе ассоциированное с ним значение. Результат проверки выводит пользователю. В случае, если активной базы данных нет, выводит пользователю соответствующее сообщение об ошибке

\item {\consolas help} – выводит пользователю список доступных команд

\item {\consolas insert} – запрашивает у пользователя ключ, потом значение, которое с ключом необходимо ассоциировать. Далее пытается внести пару ключ-значение в базу данных и сообщает пользователю о результате операции. В случае, если активной базы данных нет, выводит пользователю соответствующее сообщение об ошибке

\item {\consolas load\_db} – запрашивает у пользователя директорию и производит загрузку данных из неё. В случае неудачи сообщает об этом пользователю. Если к этому моменту уже была открыта какая-либо база данных, эта база будет закрыта без каких-либо сообщений (см. п.1)

\item {\consolas remove} – запрашивает у пользователя ключ и производит удаление ассоциированного с этим ключом значения из базы. В базе нет значения, ассоциированного с таким ключом, выводит пользователю соответствующее сообщение об ошибке. В случае, если активной базы данных нет, выводит пользователю соответствующее сообщение об ошибке

\item {\consolas run\_tests} – производит запуск ряда тестов, которые указаны в задании. Тестовая база данных будет размещена в поддиректории "test" текущей директории. Для каждого теста выводятся текущие параметры базы данных и среднее время выполнения тестируемой операции. Если на момент тестирования есть активная база данных, её состояние не изменится

\item {\consolas stats} – выводит параметры текущей базы данных, а именно: количество элементов в ней, количество цепочек, коэффициент заполнения и одной страницы в байтах. В случае, если активной базы данных нет, выводит пользователю соответствующее сообщение об ошибке
\end{enumerate}

\pagebreak
\section{Описание работы программы}
Для получения информации о скорости работы программы на больших объёмах данных, зависимости временных характеристик от времени выполнения и т.п. достаточно запустить через её основное меню выполнение тестов (команда {\consolas run\_tests}). Пример вывода данной команды:
\begin{lstlisting}[caption=,]
N: 1000 M: 10
generation: 0 ms;
page size: 336 bytes
load factor: 3.90625
ins avg: 7.881 mcs
search avg: 1.089 mcs (500 found)
del avg: 1.053 mcs

N: 1000 M: 100
generation: 0 ms;
page size: 3216 bytes
load factor: 62.5
ins avg: 7.628 mcs
search avg: 2.714 mcs (500 found)
del avg: 3.117 mcs

N: 1000 M: 1000
generation: 0 ms;
page size: 32016 bytes
load factor: 500
ins avg: 11.062 mcs
search avg: 5.336 mcs (500 found)
del avg: 3.435 mcs

N: 10000 M: 10
generation: 5 ms;
page size: 336 bytes
load factor: 4.88281
ins avg: 8.7996 mcs
search avg: 1.2788 mcs (5000 found)
del avg: 2.0078 mcs

N: 10000 M: 100
generation: 5 ms;
page size: 3216 bytes
load factor: 39.0625
ins avg: 9.3604 mcs
search avg: 1.4115 mcs (5000 found)
del avg: 2.1705 mcs

N: 10000 M: 1000
generation: 6 ms;
page size: 32016 bytes
load factor: 625
ins avg: 19.0869 mcs
search avg: 4.0938 mcs (5000 found)
del avg: 4.9218 mcs

N: 100000 M: 10
generation: 65 ms;
page size: 336 bytes
load factor: 6.10352
ins avg: 6.91499 mcs
search avg: 1.93563 mcs (50000 found)
del avg: 1.61192 mcs

N: 100000 M: 100
generation: 59 ms;
page size: 3216 bytes
load factor: 48.8281
ins avg: 8.50157 mcs
search avg: 2.12571 mcs (50000 found)
del avg: 1.81267 mcs

N: 100000 M: 1000
generation: 59 ms;
page size: 32016 bytes
load factor: 390.625
ins avg: 25.8412 mcs
search avg: 4.40859 mcs (50000 found)
del avg: 5.10267 mcs

N: 1000000 M: 10
generation: 575 ms;
page size: 336 bytes
load factor: 3.8147
ins avg: 8.54951 mcs
search avg: 2.24343 mcs (500000 found)
del avg: 1.72489 mcs

N: 1000000 M: 100
generation: 572 ms;
page size: 3216 bytes
load factor: 61.0352
ins avg: 8.0691 mcs
search avg: 2.28447 mcs (500000 found)
del avg: 1.93093 mcs

N: 1000000 M: 1000
generation: 564 ms;
page size: 32016 bytes
load factor: 488.281
ins avg: 24.8824 mcs
search avg: 4.7857 mcs (500000 found)
del avg: 5.32216 mcs
\end{lstlisting}

Как можно заметить, вставка является самой "дорогой" операцией, что и не удивительно, ведь это относят к одному из основных недостатков хеш-таблиц. Причиной этого является то, что вставка провоцирует процедуру рехеша, во время которой таблица полностью перестраивается, а так как хеш-таблица хранится во внешней памяти, это занимает ещё более значительное время, чем обычно. 

Также можно заметить, что дольше всего работает вариант, когда на одной странице находится 1000 элементов. Этому способствует сразу ряд причин: приходится выставлять очень высокую границу коэффициента заполнения, иначе каждая цепочка занимает по 32КБ дискового пространства и на последнем тесте размер базы превышает 3,5ГБ, что приводит к неприемлемо долгому выполнению следующей операции рехеша. В результате того, что на последнем тесте коэффициент заполнения достигает значения 488, выполнение всех операций замедляется, ведь каждый раз приходится выполнять поиск в массиве длинной, в среднем, 488, а это не может не сказаться на скорости работы.

Примечательным является и то, что увеличение числа элементов почти не сказывается на времени выполнения операций (для случаев с 10 и 100 элементами на странице), что, в общем-то, подтверждается теоретически, ведь алгоритмическая сложность основных операций над хеш-таблицей оценивается как $O(1+\alpha)$, где $\alpha$ – это коэффициент заполнения.
\pagebreak
\section{Заключение}
В ходе работы реализована структура данных, обладающая очень ценными характеристиками: поиск и удаление за практически константное время, при этом она занимает очень мало места в оперативной памяти и не требовательна к вычислительным ресурсам. Основной проблемой данной структуры является необходимость проведения рехеша. Так как при этом количество цепочек растет по экспоненте, со временем эта операция с одной стороны требуется всё реже и реже, а с другой – время её выполнения так же увеличивается по экспоненте, что на очень больших объемах данных может стать существенной проблемой. Интерфейс данной структуры был сделан как можно более удобным и самодокуметирующимся, т.е. для понимания того, как пользоваться тем или иным методом и как он себя будет вести достаточно знать какая у него сигнатура.

\pagebreak
\section{Список используемой литературы}
\begin{enumerate}
\item Standard C++ Library reference [Электронный ресурс]. – Режим доступа:\\ http://www.cplusplus.com/reference/

\item Кормен, Т., Лейзерсон, Ч., Ривест, Р., Штайн, К. \\Глава 11. Хеш-таблицы. // Алгоритмы: построение и анализ = Introduction to Algorithms / Под ред. И. В. Красикова. — 2-е изд. — М.: Вильямс, 2005. — 1296 с.

\end{enumerate}

\pagebreak
\section*{Приложения}
{\consolas main.cpp}
\includecode[c++, basicstyle=\scriptsize\consolas]{main.cpp}

{\consolas hash\_file\_storage.hpp}
\includecode[c++, basicstyle=\scriptsize\consolas]{hash_file_storage.hpp}

\end{document}
